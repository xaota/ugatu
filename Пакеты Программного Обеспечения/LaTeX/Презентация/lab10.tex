\documentclass{beamer}
\usepackage[english,russian]{babel}
\usepackage[utf8]{inputenc}
\usepackage{mathtext}
% Стиль презентации
\usetheme{Singapore}
\usecolortheme{default}

\begin{document}
\title{Моделирование процесса прокатки титанового прутка в пакете Deform 3D}  
\author{Ибрагимов Ринат, ОНФ ПМИ-34}
\date{УГАТУ, 2013}
% Создание заглавной страницы
\frame{\titlepage}
\begin{frame}{Содержание}
 \tableofcontents
\end{frame}
\begin{frame}{}
\transdissolve[duration=0.2]
 Описание процесса моделирования прокатки титанового прутка в программе «Deform 3D» и анализ результатов с целью дальнейшего изучения изменения зернистости  металла
\end{frame}
\section{Постановка задачи}
\begin{frame}{Постановка задачи}
 \transdissolve[duration=0.2]
 \begin{itemize}
  \item<1-> \color<1-2>{red} Изучить систему конечно-элементного моделирования DEFORM-3D, предназначенную для анализа трёхмерного течения металла при различных процессах обработки металла давлением.
  \item<2-> \color<2>{green} Пользуясь системой DEFORM смоделировать процесс прокатки титанового прутка метровой длины и оценить изменения произошедшие в его геометрии в результате процесса.
 \end{itemize}
\end{frame}
\section{Практическая часть}
\begin{frame}{Практическая часть}
\label{1.1}
В Пакете Deform 3D смоделирован механизм, состоящий из двух вращающихся валов и направляющей трубки, в которую будет помещаться пруток.

Валы вращаются таким образом, что протскивают пруток между собой, обжимая его, при этом расширяя и удлинняя.

Расстояние между валами равно пяти миллиметрам.
\end{frame}

\begin{frame}{}
\label{1.2}
\begin{figure}[h]
 \includegraphics[width=3in]{img/step52.JPG}
 \footnotesize\caption{Вид установки в момент расчёта}
\end{figure}
\end{frame}

\begin{frame}{}
\label{1.3}
\begin{figure}[h]
 \includegraphics[width=3in]{img/step207.JPG}
 \footnotesize\caption{Вид установки в конце расчёта}
\end{figure}
\end{frame}

\begin{frame}{}
\label{1.4}
\begin{figure}[h]
 \includegraphics[width=3in]{img/step207_3.JPG}
 \footnotesize\caption{Вид прутка в конце расчёта}
\end{figure}
\end{frame}

\section{Дополнительно}
\begin{frame}{}
В процессе выполнения работы таже были проверены другие материалы и температуры
\begin{figure}[h]
 \includegraphics[width=3in]{img/164_2.JPG}
 \footnotesize\caption{Алюминиевый прут}
\end{figure}
\end{frame}

\section{Заключение}
\begin{frame}{Заключение}
С помощью системы DEFORM-3D удалось \hyperlink{1.3}{смоделировать процесс прокатки титанового прута 1000x15мм}, и оценить его \hyperlink{1.4}{деформации}.
При сжатии до высоты в 7мм, происходит удлинение более чем в полтора раза, расширение приблизительно до 24мм, и изгибы, из-за которых, к сожалению, не удаётся точно измерить получившуюся геометрию средствами DEFORM.
\end{frame}

\section{Список литературы}
\begin{frame}{Список литературы}
В.С. Паршин, А.П. Карамышев, И.И. Некрасов, А.И. Пугин, А.А. Федулов
Учебное пособие под редакцией д-ра технических наук Ю.Б. Чечулина
Практическое руководство к программному компексу DEFORM-3D
Екатеринбург, УрФУ, 2010
\end{frame}

\begin{frame}{Спасибо за внимание!}
\vfill
\centering\LaTeX
\end{frame}
\end{document}