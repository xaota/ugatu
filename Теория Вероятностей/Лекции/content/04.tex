\section{Примеры некоторых вероятностных пространств}

$A$ -- открытое, либо замкнутое множество,\\
$\Omega$ -- метрическое пространство с некоторой метрикой $\rho(x,y)$.

\begin{definition}
  $\min \sigma$-алгебра, содержащая все открытые и замкнутые множества, называется \textbf{Борелевской $\sigma$-алгеброй} $\beta(x)$.
\end{definition}

\subsection{Дискретное вероятностное пространство}

$\Omega$ -- некоторое множество.\\
$B_k\cap B_{k'}=\emptyset$, при $k\ne k'$.\\
$F=\sigma(\{B_k\})$.\\
$A\in F \Leftrightarrow A=\bigcup\limits_{k'} B_{k'}$\\
$\Omega = \bigcap\limits_{k'} B_{k'}, B_k B_{k'}=\emptyset$, при $k\ne k'$.\\
$A = B_{k_1}+…+B_{k_n} \Rightarrow P(A) = P(B_{k_1})+…+P(B_{k_n})$\\
$P(B_k)=P_k, P_k \geqslant 0, \sum\limits_k P_k =1$.

\subsection{Пространство $(R, \beta (R), P)$}

$\rho(x,y) = |x-y|$ -- Евклидова метрика на $R$.\\
Однако, удобно использовать такую метрику: $\rho(x,y)=\frac{|x-y|}{1+|x-y|}.$\\
$\beta (R) - \min \sigma$-алгебра, содержащая все открытые и замкнутые множества $R$.\\
Примеры множеств в $\beta (R): (a, b), [a, b], [a, b), (a, b]$.\\
Мера на вещественной прямой задается через функцию распределения.\\
$F(x)=P((-\infty, x])$ -- функция распределения меры $P$.

\begin{enumerate}
  \item $0\leqslant F(x)\leqslant 1$

  \item $F(x)$ возрастает, т.е., $\forall x_1, x_2: x_1 \leq x_2 \Rightarrow F(x_1) \leqslant F(x_2)$.\\
    Представим множество $(-\infty, x_2]$ как сумму несовместных множеств:\\
    $(-\infty, x_2]=(-\infty, x_1]+(x_1, x_2]$, тогда $P((-\infty, x_2]) = P((-\infty, x_1])+P((x_1, x_2])$\\
    $F(x_2)=F(x_1)+P((x_1, x_2])$\\
    $F(x_2)\geqslant F(x_1)$

  \item $P((x_1, x_2]) = F(x_2)-F(x_1)$

  \item Непрерывность справа\\
    $x_n\downarrow x\Rightarrow F(x_n)\downarrow F(x)$\\
    $B_n = (-\infty, x_n]$\\
    $F(x_n)=P((-\infty, x_n])=P(B_n)$\\
    Если $x_n\downarrow x$, то
    \begin{equation}
      %\label{eqn:1}
      \tag{$\star$}
      \bigcup\limits_n B_n = B, B_n \supseteq B_{n+1}
    \end{equation}

    В силу непрерывности вероятностной меры
    \begin{displaymath}
      \lim\limits_{n\rightarrow\infty} P(B_n) = P(\bigcap\limits_n B_n)\stackrel{(\star)}{=}P((-\infty, x])
    \end{displaymath}

  \item $F(-\infty)=0, F(+\infty)=1$\\
    $F(x_n)=P(B_n); B_n=(-\infty, x_n]$\\
    $F(\infty)=\lim\limits_{n\rightarrow\infty} F(x_n)$.

    Рассмотрим два случая:
    \begin{enumerate}
      \item $x_n\downarrow -\infty$;\\
        $\bigcap\limits_n B_n=\emptyset, \lim\limits_{n\rightarrow\infty} F(x_n) = \lim\limits_{n\rightarrow\infty} P(B_n) = P(\emptyset) = 0$

      \item $x_n\downarrow +\infty$;\\
        $\bigcup\limits_n B_n=R, \lim\limits_{n\rightarrow\infty} F(x_n) = \lim\limits_{n\rightarrow\infty} P(B_n) = P(R) = 1$
    \end{enumerate}
\end{enumerate}
